
%% bare_jrnl.tex
%% V1.3
%% 2007/01/11
%% by Michael Shell
%% see http://www.michaelshell.org/
%% for current contact information.
%%
%% This is a skeleton file demonstrating the use of IEEEtran.cls
%% (requires IEEEtran.cls version 1.7 or later) with an IEEE journal paper.
%%
%% Support sites:
%% http://www.michaelshell.org/tex/ieeetran/
%% http://www.ctan.org/tex-archive/macros/latex/contrib/IEEEtran/
%% and
%% http://www.ieee.org/



% *** Authors should verify (and, if needed, correct) their LaTeX system  ***
% *** with the testflow diagnostic prior to trusting their LaTeX platform ***
% *** with production work. IEEE's font choices can trigger bugs that do  ***
% *** not appear when using other class files.                            ***
% The testflow support page is at:
% http://www.michaelshell.org/tex/testflow/


%%*************************************************************************
%% Legal Notice:
%% This code is offered as-is without any warranty either expressed or
%% implied; without even the implied warranty of MERCHANTABILITY or
%% FITNESS FOR A PARTICULAR PURPOSE! 
%% User assumes all risk.
%% In no event shall IEEE or any contributor to this code be liable for
%% any damages or losses, including, but not limited to, incidental,
%% consequential, or any other damages, resulting from the use or misuse
%% of any information contained here.
%%
%% All comments are the opinions of their respective authors and are not
%% necessarily endorsed by the IEEE.
%%
%% This work is distributed under the LaTeX Project Public License (LPPL)
%% ( http://www.latex-project.org/ ) version 1.3, and may be freely used,
%% distributed and modified. A copy of the LPPL, version 1.3, is included
%% in the base LaTeX documentation of all distributions of LaTeX released
%% 2003/12/01 or later.
%% Retain all contribution notices and credits.
%% ** Modified files should be clearly indicated as such, including  **
%% ** renaming them and changing author support contact information. **
%%
%% File list of work: IEEEtran.cls, IEEEtran_HOWTO.pdf, bare_adv.tex,
%%                    bare_conf.tex, bare_jrnl.tex, bare_jrnl_compsoc.tex
%%*************************************************************************

% Note that the a4paper option is mainly intended so that authors in
% countries using A4 can easily print to A4 and see how their papers will
% look in print - the typesetting of the document will not typically be
% affected with changes in paper size (but the bottom and side margins will).
% Use the testflow package mentioned above to verify correct handling of
% both paper sizes by the user's LaTeX system.
%
% Also note that the "draftcls" or "draftclsnofoot", not "draft", option
% should be used if it is desired that the figures are to be displayed in
% draft mode.
%
\documentclass[journal]{IEEEtran}
\usepackage{blindtext}
\usepackage{graphicx}

% Some very useful LaTeX packages include:
% (uncomment the ones you want to load)


% *** MISC UTILITY PACKAGES ***
%
%\usepackage{ifpdf}
% Heiko Oberdiek's ifpdf.sty is very useful if you need conditional
% compilation based on whether the output is pdf or dvi.
% usage:
% \ifpdf
%   % pdf code
% \else
%   % dvi code
% \fi
% The latest version of ifpdf.sty can be obtained from:
% http://www.ctan.org/tex-archive/macros/latex/contrib/oberdiek/
% Also, note that IEEEtran.cls V1.7 and later provides a builtin
% \ifCLASSINFOpdf conditional that works the same way.
% When switching from latex to pdflatex and vice-versa, the compiler may
% have to be run twice to clear warning/error messages.






% *** CITATION PACKAGES ***
%
%\usepackage{cite}
% cite.sty was written by Donald Arseneau
% V1.6 and later of IEEEtran pre-defines the format of the cite.sty package
% \cite{} output to follow that of IEEE. Loading the cite package will
% result in citation numbers being automatically sorted and properly
% "compressed/ranged". e.g., [1], [9], [2], [7], [5], [6] without using
% cite.sty will become [1], [2], [5]--[7], [9] using cite.sty. cite.sty's
% \cite will automatically add leading space, if needed. Use cite.sty's
% noadjust option (cite.sty V3.8 and later) if you want to turn this off.
% cite.sty is already installed on most LaTeX systems. Be sure and use
% version 4.0 (2003-05-27) and later if using hyperref.sty. cite.sty does
% not currently provide for hyperlinked citations.
% The latest version can be obtained at:
% http://www.ctan.org/tex-archive/macros/latex/contrib/cite/
% The documentation is contained in the cite.sty file itself.






% *** GRAPHICS RELATED PACKAGES ***
%
\ifCLASSINFOpdf
  % \usepackage[pdftex]{graphicx}
  % declare the path(s) where your graphic files are
  % \graphicspath{{../pdf/}{../jpeg/}}
  % and their extensions so you won't have to specify these with
  % every instance of \includegraphics
  % \DeclareGraphicsExtensions{.pdf,.jpeg,.png}
\else
  % or other class option (dvipsone, dvipdf, if not using dvips). graphicx
  % will default to the driver specified in the system graphics.cfg if no
  % driver is specified.
  % \usepackage[dvips]{graphicx}
  % declare the path(s) where your graphic files are
  % \graphicspath{{../eps/}}
  % and their extensions so you won't have to specify these with
  % every instance of \includegraphics
  % \DeclareGraphicsExtensions{.eps}
\fi
% graphicx was written by David Carlisle and Sebastian Rahtz. It is
% required if you want graphics, photos, etc. graphicx.sty is already
% installed on most LaTeX systems. The latest version and documentation can
% be obtained at: 
% http://www.ctan.org/tex-archive/macros/latex/required/graphics/
% Another good source of documentation is "Using Imported Graphics in
% LaTeX2e" by Keith Reckdahl which can be found as epslatex.ps or
% epslatex.pdf at: http://www.ctan.org/tex-archive/info/
%
% latex, and pdflatex in dvi mode, support graphics in encapsulated
% postscript (.eps) format. pdflatex in pdf mode supports graphics
% in .pdf, .jpeg, .png and .mps (metapost) formats. Users should ensure
% that all non-photo figures use a vector format (.eps, .pdf, .mps) and
% not a bitmapped formats (.jpeg, .png). IEEE frowns on bitmapped formats
% which can result in "jaggedy"/blurry rendering of lines and letters as
% well as large increases in file sizes.
%
% You can find documentation about the pdfTeX application at:
% http://www.tug.org/applications/pdftex





% *** MATH PACKAGES ***
%
%\usepackage[cmex10]{amsmath}
% A popular package from the American Mathematical Society that provides
% many useful and powerful commands for dealing with mathematics. If using
% it, be sure to load this package with the cmex10 option to ensure that
% only type 1 fonts will utilized at all point sizes. Without this option,
% it is possible that some math symbols, particularly those within
% footnotes, will be rendered in bitmap form which will result in a
% document that can not be IEEE Xplore compliant!
%
% Also, note that the amsmath package sets \interdisplaylinepenalty to 10000
% thus preventing page breaks from occurring within multiline equations. Use:
%\interdisplaylinepenalty=2500
% after loading amsmath to restore such page breaks as IEEEtran.cls normally
% does. amsmath.sty is already installed on most LaTeX systems. The latest
% version and documentation can be obtained at:
% http://www.ctan.org/tex-archive/macros/latex/required/amslatex/math/





% *** SPECIALIZED LIST PACKAGES ***
%
%\usepackage{algorithmic}
% algorithmic.sty was written by Peter Williams and Rogerio Brito.
% This package provides an algorithmic environment fo describing algorithms.
% You can use the algorithmic environment in-text or within a figure
% environment to provide for a floating algorithm. Do NOT use the algorithm
% floating environment provided by algorithm.sty (by the same authors) or
% algorithm2e.sty (by Christophe Fiorio) as IEEE does not use dedicated
% algorithm float types and packages that provide these will not provide
% correct IEEE style captions. The latest version and documentation of
% algorithmic.sty can be obtained at:
% http://www.ctan.org/tex-archive/macros/latex/contrib/algorithms/
% There is also a support site at:
% http://algorithms.berlios.de/index.html
% Also of interest may be the (relatively newer and more customizable)
% algorithmicx.sty package by Szasz Janos:
% http://www.ctan.org/tex-archive/macros/latex/contrib/algorithmicx/




% *** ALIGNMENT PACKAGES ***
%
%\usepackage{array}
% Frank Mittelbach's and David Carlisle's array.sty patches and improves
% the standard LaTeX2e array and tabular environments to provide better
% appearance and additional user controls. As the default LaTeX2e table
% generation code is lacking to the point of almost being broken with
% respect to the quality of the end results, all users are strongly
% advised to use an enhanced (at the very least that provided by array.sty)
% set of table tools. array.sty is already installed on most systems. The
% latest version and documentation can be obtained at:
% http://www.ctan.org/tex-archive/macros/latex/required/tools/


%\usepackage{mdwmath}
%\usepackage{mdwtab}
% Also highly recommended is Mark Wooding's extremely powerful MDW tools,
% especially mdwmath.sty and mdwtab.sty which are used to format equations
% and tables, respectively. The MDWtools set is already installed on most
% LaTeX systems. The lastest version and documentation is available at:
% http://www.ctan.org/tex-archive/macros/latex/contrib/mdwtools/


% IEEEtran contains the IEEEeqnarray family of commands that can be used to
% generate multiline equations as well as matrices, tables, etc., of high
% quality.


%\usepackage{eqparbox}
% Also of notable interest is Scott Pakin's eqparbox package for creating
% (automatically sized) equal width boxes - aka "natural width parboxes".
% Available at:
% http://www.ctan.org/tex-archive/macros/latex/contrib/eqparbox/





% *** SUBFIGURE PACKAGES ***
\usepackage[tight,footnotesize]{subfigure}
% subfigure.sty was written by Steven Douglas Cochran. This package makes it
% easy to put subfigures in your figures. e.g., "Figure 1a and 1b". For IEEE
% work, it is a good idea to load it with the tight package option to reduce
% the amount of white space around the subfigures. subfigure.sty is already
% installed on most LaTeX systems. The latest version and documentation can
% be obtained at:
% http://www.ctan.org/tex-archive/obsolete/macros/latex/contrib/subfigure/
% subfigure.sty has been superceeded by subfig.sty.



%\usepackage[caption=false]{caption}
%\usepackage[font=footnotesize]{subfig}
% subfig.sty, also written by Steven Douglas Cochran, is the modern
% replacement for subfigure.sty. However, subfig.sty requires and
% automatically loads Axel Sommerfeldt's caption.sty which will override
% IEEEtran.cls handling of captions and this will result in nonIEEE style
% figure/table captions. To prevent this problem, be sure and preload
% caption.sty with its "caption=false" package option. This is will preserve
% IEEEtran.cls handing of captions. Version 1.3 (2005/06/28) and later 
% (recommended due to many improvements over 1.2) of subfig.sty supports
% the caption=false option directly:
%\usepackage[caption=false,font=footnotesize]{subfig}
%
% The latest version and documentation can be obtained at:
% http://www.ctan.org/tex-archive/macros/latex/contrib/subfig/
% The latest version and documentation of caption.sty can be obtained at:
% http://www.ctan.org/tex-archive/macros/latex/contrib/caption/




% *** FLOAT PACKAGES ***
%
\usepackage{fixltx2e}
% fixltx2e, the successor to the earlier fix2col.sty, was written by
% Frank Mittelbach and David Carlisle. This package corrects a few problems
% in the LaTeX2e kernel, the most notable of which is that in current
% LaTeX2e releases, the ordering of single and double column floats is not
% guaranteed to be preserved. Thus, an unpatched LaTeX2e can allow a
% single column figure to be placed prior to an earlier double column
% figure. The latest version and documentation can be found at:
% http://www.ctan.org/tex-archive/macros/latex/base/



%\usepackage{stfloats}
% stfloats.sty was written by Sigitas Tolusis. This package gives LaTeX2e
% the ability to do double column floats at the bottom of the page as well
% as the top. (e.g., "\begin{figure*}[!b]" is not normally possible in
% LaTeX2e). It also provides a command:
%\fnbelowfloat
% to enable the placement of footnotes below bottom floats (the standard
% LaTeX2e kernel puts them above bottom floats). This is an invasive package
% which rewrites many portions of the LaTeX2e float routines. It may not work
% with other packages that modify the LaTeX2e float routines. The latest
% version and documentation can be obtained at:
% http://www.ctan.org/tex-archive/macros/latex/contrib/sttools/
% Documentation is contained in the stfloats.sty comments as well as in the
% presfull.pdf file. Do not use the stfloats baselinefloat ability as IEEE
% does not allow \baselineskip to stretch. Authors submitting work to the
% IEEE should note that IEEE rarely uses double column equations and
% that authors should try to avoid such use. Do not be tempted to use the
% cuted.sty or midfloat.sty packages (also by Sigitas Tolusis) as IEEE does
% not format its papers in such ways.


%\ifCLASSOPTIONcaptionsoff
%  \usepackage[nomarkers]{endfloat}
% \let\MYoriglatexcaption\caption
% \renewcommand{\caption}[2][\relax]{\MYoriglatexcaption[#2]{#2}}
%\fi
% endfloat.sty was written by James Darrell McCauley and Jeff Goldberg.
% This package may be useful when used in conjunction with IEEEtran.cls'
% captionsoff option. Some IEEE journals/societies require that submissions
% have lists of figures/tables at the end of the paper and that
% figures/tables without any captions are placed on a page by themselves at
% the end of the document. If needed, the draftcls IEEEtran class option or
% \CLASSINPUTbaselinestretch interface can be used to increase the line
% spacing as well. Be sure and use the nomarkers option of endfloat to
% prevent endfloat from "marking" where the figures would have been placed
% in the text. The two hack lines of code above are a slight modification of
% that suggested by in the endfloat docs (section 8.3.1) to ensure that
% the full captions always appear in the list of figures/tables - even if
% the user used the short optional argument of \caption[]{}.
% IEEE papers do not typically make use of \caption[]'s optional argument,
% so this should not be an issue. A similar trick can be used to disable
% captions of packages such as subfig.sty that lack options to turn off
% the subcaptions:
% For subfig.sty:
% \let\MYorigsubfloat\subfloat
% \renewcommand{\subfloat}[2][\relax]{\MYorigsubfloat[]{#2}}
% For subfigure.sty:
% \let\MYorigsubfigure\subfigure
% \renewcommand{\subfigure}[2][\relax]{\MYorigsubfigure[]{#2}}
% However, the above trick will not work if both optional arguments of
% the \subfloat/subfig command are used. Furthermore, there needs to be a
% description of each subfigure *somewhere* and endfloat does not add
% subfigure captions to its list of figures. Thus, the best approach is to
% avoid the use of subfigure captions (many IEEE journals avoid them anyway)
% and instead reference/explain all the subfigures within the main caption.
% The latest version of endfloat.sty and its documentation can obtained at:
% http://www.ctan.org/tex-archive/macros/latex/contrib/endfloat/
%
% The IEEEtran \ifCLASSOPTIONcaptionsoff conditional can also be used
% later in the document, say, to conditionally put the References on a 
% page by themselves.





% *** PDF, URL AND HYPERLINK PACKAGES ***
%
%\usepackage{url}
% url.sty was written by Donald Arseneau. It provides better support for
% handling and breaking URLs. url.sty is already installed on most LaTeX
% systems. The latest version can be obtained at:
% http://www.ctan.org/tex-archive/macros/latex/contrib/misc/
% Read the url.sty source comments for usage information. Basically,
% \url{my_url_here}.





% *** Do not adjust lengths that control margins, column widths, etc. ***
% *** Do not use packages that alter fonts (such as pslatex).         ***
% There should be no need to do such things with IEEEtran.cls V1.6 and later.
% (Unless specifically asked to do so by the journal or conference you plan
% to submit to, of course. )


% correct bad hyphenation here
\hyphenation{op-tical net-works semi-conduc-tor}

%Change captions
\usepackage{subcaption}
\usepackage[font=footnotesize,labelfont=normalfont ]{caption}
\captionsetup[table]{name=Table,belowskip=3pt,aboveskip=4pt}
\renewcommand{\thetable}{\arabic{table}}

%Allow for pbox formatting in table
\usepackage{pbox}

\begin{document}
%

% paper title
% can use linebreaks \\ within to get better formatting as desired
\title{NYC Real Estate Price Prediction}
%
%
% author names and IEEE memberships
% note positions of commas and nonbreaking spaces ( ~ ) LaTeX will not break
% a structure at a ~ so this keeps an author's name from being broken across
% two lines.
% use \thanks{} to gain access to the first footnote area
% a separate \thanks must be used for each paragraph as LaTeX2e's \thanks
% was not built to handle multiple paragraphs
%

\author{Ben~Jakubowski~and~Haonan~Zhou}

% note the % following the last \IEEEmembership and also \thanks - 
% these prevent an unwanted space from occurring between the last author name
% and the end of the author line. i.e., if you had this:
% 
% \author{....lastname \thanks{...} \thanks{...} }
%                     ^------------^------------^----Do not want these spaces!
%
% a space would be appended to the last name and could cause every name on that
% line to be shifted left slightly. This is one of those "LaTeX things". For
% instance, "\textbf{A} \textbf{B}" will typeset as "A B" not "AB". To get
% "AB" then you have to do: "\textbf{A}\textbf{B}"
% \thanks is no different in this regard, so shield the last } of each \thanks
% that ends a line with a % and do not let a space in before the next \thanks.
% Spaces after \IEEEmembership other than the last one are OK (and needed) as
% you are supposed to have spaces between the names. For what it is worth,
% this is a minor point as most people would not even notice if the said evil
% space somehow managed to creep in.



% The paper headers
\markboth{NYU CDS CAPSTONE PROJECT 2016}%
{Shell \MakeLowercase{\textit{et al.}}: Bare Demo of IEEEtran.cls for Journals}
% The only time the second header will appear is for the odd numbered pages
% after the title page when using the twoside option.
% 
% *** Note that you probably will NOT want to include the author's ***
% *** name in the headers of peer review papers.                   ***
% You can use \ifCLASSOPTIONpeerreview for conditional compilation here if
% you desire.




% If you want to put a publisher's ID mark on the page you can do it like
% this:
%\IEEEpubid{0000--0000/00\$00.00~\copyright~2007 IEEE}
% Remember, if you use this you must call \IEEEpubidadjcol in the second
% column for its text to clear the IEEEpubid mark.



% use for special paper notices
%\IEEEspecialpapernotice{(Invited Paper)}




% make the title area
\maketitle


\begin{abstract}
%\boldmath
We built supervised learning models to predict NYC property prices using data scraped from StreetEasy's website. Our best model for this problem was XGboost with hyper-parameters optimized using grid search and cross-validation. The final model had an out-of-sample median absolute percentage error of 5\%, which is comparable to the results of Zillow's real estate price prediction model.
\end{abstract}
% IEEEtran.cls defaults to using nonbold math in the Abstract.
% This preserves the distinction between vectors and scalars. However,
% if the journal you are submitting to favors bold math in the abstract,
% then you can use LaTeX's standard command \boldmath at the very start
% of the abstract to achieve this. Many IEEE journals frown on math
% in the abstract anyway.

% Note that keywords are not normally used for peerreview papers.
\begin{IEEEkeywords}
NYC Real Estate, StreetEasy, XGBoost
\end{IEEEkeywords}






% For peer review papers, you can put extra information on the cover
% page as needed:
% \ifCLASSOPTIONpeerreview
% \begin{center} \bfseries EDICS Category: 3-BBND \end{center}
% \fi
%
% For peerreview papers, this IEEEtran command inserts a page break and
% creates the second title. It will be ignored for other modes.
\IEEEpeerreviewmaketitle



\section{Introduction}

The value of real estate units are most often understood by examining sales of real estate comps (comparables). These comps are used by property owners to assess the market values of their existing units and property developers to decide whether it is profitable to construct new units. Very often, much of the process of determining real estate comps are conducted in a qualitative and haphazard method that only uses basic information such as price per square foot of recent sales in similar neighborhoods. Moreover, often this information is gather informally (through professional networks) as a limited sample of known recent sales \cite{pagourtzi2003real}. 

In this project, we used an automated, systematic, and extensible machine learning approach to determine real estate values. We developed a prediction model of NYC real estate prices using supervised learning models instead of relying on expert intuition and a priori definitions of comparability. We trained various models using features engineered from publicly available property sales and listings records, and we evaluated the models by examining various percentiles of out-of-sample absolute percent error in predicted price. This objective is also used by the real estate website Zillow to evaluate the accuracy of their price predictions, and thus it can be viewed as the industry standard evaluation metric.

\section{Data Collection}

When initially approaching this predictive modeling problem (predicting real estate prices in NYC), we began by exploring civic datasets related to NYC real estate; these datasets included the Department of Finance Annualized Sales data, which provides records of property sales for each year, and the PLUTO dataset, which provides information about tax lots (including features describing building age and assessed tax value). Unfortunately, these data were very feature poor; when predicting property values, current modeling approaches often use features like the number of beds/baths and the square footage \cite{pagourtzi2003real}. Since civic datasets don't include these or other property-level features, we sought out alternative datasets.

Ultimately, we decided to make our primary dataset sale pages from StreetEasy, an online real estate database with a wealth of NYC property listings and associated features. We used the web-crawling tool Scrapy in conjunction with the Tor network to download the HTML source of all historical listings available on the StreetEasy website by systematically querying all sales pages (fetching pages indexed by integers from 1-1400000). This yielded 516k entries of properties listings dating from 1997 to the present day (November 2016). We proceeded to use the BeautifulSoup python package to extract data from various HTML tags on the sales page based on our knowledge of the web-page structure. The set of raw features that we obtained from the StreetEasy property listings include the following:

\begin{itemize}
    \item Sale or final listing price
    \item Sale or delisting date
    \item Final status of entry (sold, delisted, etc)
    \item Property size (square-footage, number of bed and baths)
    \item Property type (condo, co-op, house, etc)
    \item Monthly costs (taxes, maintenance, etc)
    \item Building data (amenities, number of units, age, etc)
    \item Distance to public transportation (MTA, PATH, etc)
    \item Textual description of listing (human written)
    \item GPS coordinates of property
\end{itemize}

After the raw features were extracted, we chose to discard records of sales outside the five boroughs (a large number of sales were reported in New Jersey and the greater NYC metro region). Then we immediately proceeded to split the sales records into 80-20 train-test data sets ensure our final model evaluation provided an accurate estimate of generalization error. All data points collected from StreetEasy contain a  final sale or listing price, which is required for the construction the target variable in our supervised learning models. Nevertheless, we still performed some minimal data filtering in order to obtain a data set that is more suitable for our modeling goals. We removed all records without an associated GPS coordinate because these are mostly spurious or misnamed listings with incomplete or incorrect address information. We also removed entries with price outside of the 5th to 95th percentile price band of the training set, as these outliers have erroneous price or represent special properties with different data generating processes (i.e. the market for sales of buildings is anticipated to operate differently than the market for sales of condos). The size of the training and test sets are reduced to 286k and 71k records respectively after the data is filtered.

\section{Feature Engineering}

The raw data extracted from Streeteasy is abundant in missing entries. We used standard techniques to eliminate these NaN entries before using the data for our models:
\begin{itemize}
    \item Filled missing categorical features with dummy label
    \item Filled missing numerical features with mean value
\end{itemize}
In addition, we transformed raw features and constructed new features using various techniques:
\begin{itemize}
    \item Used one-hot-encoding to represent categorical features
    \item Inferred neighborhood, borough, and community district from property GPS coordinates and constructed new geographical indicators from this analysis 
    \item Constructed comp features for listing price and unit size from comparable units in (i) the same building and (ii) the same neighborhood
    \item Modeled listings description text field using SLDA to generate topic weights as features
\end{itemize}

\subsection*{LDA Analysis}

While the scraped HTML pages provided many features that were useable with minimal processing (ex: number of bedrooms), one of the major components of the sale pages was a human written property description. After reviewing a number of these descriptions, we hypothesized they reflected several latent topics with semantics related to property aesthetics and amenities (for example, luxury condos in large buildings versus prewar buildings in quiet residential neighborhoods). Thus, we proceeded to
\begin{itemize}
    \item Further split the training data into training and validation subsets
    \item Run a grid search over $K$ in [0,4,5,6,7,8,9,10,15,20,25] topics, for two types of topic models: Latent Dirichlet Allocation \cite{blei2003latent} and supervised Latent Dirichlet Allocation \cite{mcauliffe2008supervised}
\end{itemize}

\begin{figure}
    \centering
    \includegraphics[width=3in,height=2.5in,clip,keepaspectratio]{learning_curve.png}
    \caption{Validation set MSE for sLDA and LDA topic models for grid search over $K$}
    \label{fig:LDA_grid_search}
\end{figure}

The learning curves for the grid search are shown in Fig. \ref{fig:LDA_grid_search}. 
Based on these results, we chose to use sLDA with $K = 10$ topics. This was based on the observations that (i) additional topics did not improve validation set performance, and (ii) linear modeling using sLDA topics performed as well as non-linear modeling using LDA topics, so non-linear modeling with sLDA topics would perform at least as well. To further validate our use of this topic representation, we compared the learned topic weights to the top 5 words for each topic (Fig. \ref{fig:topic_weights}), and observed higher topic weights (i.e. higher predicted price) clearly were associated with words related to higher-end properties. For reference, note the topic weights can be interpreted as the predicted log(price) for a property with a one-hot topic vector.

Additionally, we also aggregated properties by neighborhood (specific neighborhood tabulation area or NTA, a geographic division constructed by the city) and mapped NTA-level median topic weights for each topic. As shown in the sample map (Fig. \ref{fig:topic_map}), the geographic clustering of topics by neighborhood further support the use of the sLDA topics in our predictive model.

\begin{figure}
    \centering
    \includegraphics[width=3in,height=2.5in,clip,keepaspectratio]{topic_weights.jpeg}
    \caption{Topic weights are learned along with topics in sLDA, and the top five words for each topic and the topic's associated weight are shown.}
    \label{fig:topic_weights}
\end{figure}

\begin{figure}
    \centering
    \includegraphics[width=3in,height=2.5in,clip,keepaspectratio]{topic_7_map.png}
    \caption{Properties were aggregated by neighborhood, and median topic weights for each topic were mapped. This map shows median topic weights for topic 7: designed loft properties.}
    \label{fig:topic_map}
\end{figure}

\section{Modeling}
 
\subsection{Linear modeling}

Our baseline model for the regression problem was a standard MSE-loss elastic-net model trained using cross-validation to determine optimal hyper-parameters. From the cross-validation curve in Fig. \ref{fig:elastic_net}, we see that the model never enters a region of high variance even as the regularization strength is reduced significantly. This suggests that the linear model is highly biased, and a more expressive non-linear model would be more suitable for the price prediction problem.

\begin{figure}[h!]
    \includegraphics[width=3in,height=2.5in,clip,keepaspectratio]{elastic-net-cv}
    \caption{Cross validation MSE of elastic-net model with various l1/l2 regularization ratio and regularization strength (alpha).}
    \label{fig:elastic_net}
\end{figure}

We also tried to fit the elastic-net model for sub-regions of NYC (boroughs, community districts, and neighborhoods). Although this yielded better evaluation scores, the improvement is not significant compared to the city-wide model. Table \ref{tab:elastic_net}. summarizes the model evaluation results for the city-wide and four borough-wide models in comparison with the industry benchmark from Zillow. It is clear that there is still a lot more to improve before our model can be competitive with the benchmark.

\begin{table}[h!]
    \caption*{Borough-Level Elastic Net Models}
    \centering
    \begin{tabular}{ | l | l | }
    \hline
    Model & Median Absolute Percentage Error  \\ \hline
    All NYC & 21.2   \\ \hline
    Manhattan & 17.91  \\ \hline
    Brooklyn & 20.58   \\ \hline
    Queens & 18.59  \\ \hline
    Bronx & 21.85   \\ \hline
    Zillow Benchmark & 5.2   \\ \hline 
    \end{tabular}
    \caption{Test set evaluation errors of normalized prices using elastic-net model with regularization coefficient and l1/l2 regularization ratio chosen by 3-fold cross-validation.}
    \label{tab:elastic_net} 
\end{table}

\subsection{Non-linear Modeling}
\begin{table}[h!]
    \caption*{Random Forest vs. XGBoost}
    \centering
    \begin{tabular}{ | l | l | l | l | }
    \hline
    Model & hyper-parameter Search Space & MSE & $R^2$  \\ \hline
    RF & \pbox{10cm}{First search: Tree depth - [4,6,8,10] \\Second search: Learning rate - [0.01,0.1,0.3,0.7]} & 0.0456 & 0.934 \\ \hline
    XGBoost & \pbox{10cm}{Tree depth - [4,8] \\ Number of Trees - [50,100,200,400]\\Number of Features - [P, sqrt(P)]} & 0.196 & 0.720 \\ \hline
    \end{tabular}
    \caption{Test set errors following initial hyper-parameter optimization for Random Forests and XGBoost} \label{tab:errors_rf_XGBoost} 
\end{table}

Since the linear models demonstrated significant bias, we proceeded to explore two approaches to city-wide non-linear modeling: random forests and tree boosting with XGBoost \cite{chen2016xgboost}. The first round of hyper-parameter optimization revealed that random forests underperformed compared to XGBoost, so subsequent tuning efforts focusing on the XGBoost model. These initial RF and XGBoost results in shown in Table \ref{tab:errors_rf_XGBoost}.


\begin{figure}[h!]
\begin{subfigure}
  \centering
  \includegraphics[width=3in,height=2.5in,clip,keepaspectratio]{XGB_model_3_depth_learning_lc.png}
\end{subfigure}
\begin{subfigure}
  \centering
  \includegraphics[width=3in,height=2.5in,clip,keepaspectratio]{XGB_model_3_num_round_eta_lc.png}
\end{subfigure}
\caption{Learning curves for XGBoost hyper-parameter optimization. To constrain runtime of hyper-parameter optimization, we first used gird search to identify the optimal tree depth on 100 rounds of boosting, then optimized the learning rate given this optimal depth}
\label{fig:xgb_learning_curves}
\end{figure}


Since XGBoost was clearly outperforming the random forest (though further tuning of the random forest could potentially have produced further gains), subsequent experiments were conducted to further optimize the XGBoost model. As shown in Fig. \ref{fig:xgb_learning_curves}, these experiments included testing:
\begin{itemize}
\item Loss functions: (i) RMSE, (ii) Root Mean Squared Percent Error (RMSPE), implemented as a custom less function
\item Target transformations: (i) Scaled, (ii) log-transformed and normalized. 
\item Learning rate: Grid search over [0.01,0.1,0.3,0.7]
\item Max tree depth: Grid search over [4,6,8,10]
\end{itemize}

Based on these experiments, the final optimized model was 1000 rounds of boosting, a maximum tree depth of 10, a learning rate of 0.1, with the objective of RMSE on the log-transformed and normalized price. Finally, using the optimal XGBoost configuration, the model was refit to the entire training set. Feature importance is shown for the 10 most important features in the final XGBoost model in Fig. \ref{fig:feature_import}.

\begin{figure}[!h]
    \centering
    \includegraphics[width=3in,height=2.5in,clip,keepaspectratio]{XGB_model_3_feature_importance.png}
    \caption{Feature importance for final model}
    \label{fig:feature_import}
\end{figure}
% needed in second column of first page if using \IEEEpubid
%\IEEEpubidadjcol

% An example of a floating figure using the graphicx package.
% Note that \label must occur AFTER (or within) \caption.
% For figures, \caption should occur after the \includegraphics.
% Note that IEEEtran v1.7 and later has special internal code that
% is designed to preserve the operation of \label within \caption
% even when the captionsoff option is in effect. However, because
% of issues like this, it may be the safest practice to put all your
% \label just after \caption rather than within \caption{}.
%
% Reminder: the "draftcls" or "draftclsnofoot", not "draft", class
% option should be used if it is desired that the figures are to be
% displayed while in draft mode.
%
%\begin{figure}[!t]
%\centering
%\includegraphics[width=2.5in]{myfigure}
% where an .eps filename suffix will be assumed under latex, 
% and a .pdf suffix will be assumed for pdflatex; or what has been declared
% via \DeclareGraphicsExtensions.
%\caption{Simulation Results}
%\label{fig_sim}
%\end{figure}

% Note that IEEE typically puts floats only at the top, even when this
% results in a large percentage of a column being occupied by floats.


% An example of a double column floating figure using two subfigures.
% (The subfig.sty package must be loaded for this to work.)
% The subfigure \label commands are set within each subfloat command, the
% \label for the overall figure must come after \caption.
% \hfil must be used as a separator to get equal spacing.
% The subfigure.sty package works much the same way, except \subfigure is
% used instead of \subfloat.
%
%\begin{figure*}[!t]
%\centerline{\subfloat[Case I]\includegraphics[width=2.5in]{subfigcase1}%
%\label{fig_first_case}}
%\hfil
%\subfloat[Case II]{\includegraphics[width=2.5in]{subfigcase2}%
%\label{fig_second_case}}}
%\caption{Simulation results}
%\label{fig_sim}
%\end{figure*}
%
% Note that often IEEE papers with subfigures do not employ subfigure
% captions (using the optional argument to \subfloat), but instead will
% reference/describe all of them (a), (b), etc., within the main caption.


% An example of a floating table. Note that, for IEEE style tables, the 
% \caption command should come BEFORE the table. Table text will default to
% \footnotesize as IEEE normally uses this smaller font for tables.
% The \label must come after \caption as always.
%
%\begin{table}[!t]
%% increase table row spacing, adjust to taste
%\renewcommand{\arraystretch}{1.3}
% if using array.sty, it might be a good idea to tweak the value of
% \extrarowheight as needed to properly center the text within the cells
%\caption{An Example of a Table}
%\label{table_example}
%\centering
%% Some packages, such as MDW tools, offer better commands for making tables
%% than the plain LaTeX2e tabular which is used here.
%\begin{tabular}{|c||c|}
%\hline
%One & Two\\
%\hline
%Three & Four\\
%\hline
%\end{tabular}
%\end{table}


% Note that IEEE does not put floats in the very first column - or typically
% anywhere on the first page for that matter. Also, in-text middle ("here")
% positioning is not used. Most IEEE journals use top floats exclusively.
% Note that, LaTeX2e, unlike IEEE journals, places footnotes above bottom
% floats. This can be corrected via the \fnbelowfloat command of the
% stfloats package.

\begin{figure}
    \centering
    \includegraphics[width=5in,height=3.5in,clip,keepaspectratio]{hist_zillow.png}
    \caption{Final model performance compared to Zillow's deployed NYC price prediction model}
    \label{fig:hist_zillow}
\end{figure}

\section{Future Work}

We went through a tremendous amount of effort to obtain a large amount of high quality data for supervised learning problem, but there is still scope to improve the quality of the data. The target variable that we used for the supervised learning problem was a price scraped from the StreetEasy sales pages. In many cases, this number is the not the final sale price (such as when a listing doesn't terminate in a sale), so there is some error in our target variable and the actual market value of the property.

One aspect of the project that we did not address in detail was the date associated with each listing. We used the final listing date as a feature for our models, but we simply treated it as a categorical feature with each year being a distinct label. There are potential improvements that can be made with the prediction accuracy if we take the continuous nature of date into account and construct more complicated features to represent changes over time.

In addition to more expressive and robust features, we can also improve our model by increasing the size of our training data set. The StreetEasy NYC listings are primarily focused on Manhattan and Brooklyn apartment style properties, so the model performs poorly on properties in other boroughs. The highly non-linear nature of the XGBoost algorithm also means that generalization error would reduce as the size of the training data set is increased. One conceptually straightforward method of getting more is to scrape other real estate webpages such as Trulia and Zillow. We can reuse many components of our StreetEasy data collection pipeline for this task, but removing duplicate entries from multiple sources and merging features in a consistent manner is a task that would require a great amount of effort

In terms of modeling, the XGBoost algorithm is extremely computationally expensive, and while we have converged on a good set of hyper-parameters, we cannot guarantee that it is optimal or even approximately optimal. Future work could be done to setup a pipeline to perform model selection in a parallel computing environment so that more model configurations can be tried within a tractable amount of time.

\section{Conclusion}

As shown in Fig. \ref{fig:hist_zillow}, our final predictive model achieved a median out-of-sample absolute percent error of 5.0\%, identical to that achieved by Zillow's NYC zestimate model. Hence, based on this industry standard evaluation metric, we conclude we learned a predictive model that is competitive with the model deployed for NYC by one of the largest online real estate sites. Importantly, our thresholding rule (training on sale records between the $5^{th}$ and $95^{th}$ percentiles) also resulted in us only making predictions on $\sim 90\%$ of the records with valid addresses, comparable to the percentage of Zillow.



% if have a single appendix:
%\appendix[Proof of the Zonklar Equations]
% or
%\appendix  % for no appendix heading
% do not use \section anymore after \appendix, only \section*
% is possibly needed

% use appendices with more than one appendix
% then use \section to start each appendix
% you must declare a \section before using any
% \subsection or using \label (\appendices by itself
% starts a section numbered zero.)
%



% Can use something like this to put references on a page
% by themselves when using endfloat and the captionsoff option.
\ifCLASSOPTIONcaptionsoff
  \newpage
\fi



% trigger a \newpage just before the given reference
% number - used to balance the columns on the last page
% adjust value as needed - may need to be readjusted if
% the document is modified later
%\IEEEtriggeratref{8}
% The "triggered" command can be changed if desired:
%\IEEEtriggercmd{\enlargethispage{-5in}}

% references section

% can use a bibliography generated by BibTeX as a .bbl file
% BibTeX documentation can be easily obtained at:
% http://www.ctan.org/tex-archive/biblio/bibtex/contrib/doc/
% The IEEEtran BibTeX style support page is at:
% http://www.michaelshell.org/tex/ieeetran/bibtex/
\bibliographystyle{IEEEtran}
\bibliography{refs}
%
% <OR> manually copy in the resultant .bbl file
% set second argument of \begin to the number of references
% (used to reserve space for the reference number labels box)
%\begin{thebibliography}{1}

%\bibitem{IEEEhowto:kopka}
%H.~Kopka and P.~W. Daly, \emph{A Guide to \LaTeX}, 3rd~ed.\hskip 1em plus
%  0.5em minus 0.4em\relax Harlow, England: Addison-Wesley, 1999.

%\end{thebibliography}

% biography section
% 
% If you have an EPS/PDF photo (graphicx package needed) extra braces are
% needed around the contents of the optional argument to biography to prevent
% the LaTeX parser from getting confused when it sees the complicated
% \includegraphics command within an optional argument. (You could create
% your own custom macro containing the \includegraphics command to make things
% simpler here.)
%\begin{biography}[{\includegraphics[width=1in,height=1.25in,clip,keepaspectratio]{mshell}}]{Michael Shell}
% or if you just want to reserve a space for a photo:

%\begin{IEEEbiography}[{\includegraphics[width=1in,height=1.25in,clip,keepaspectratio]{picture}}]{John Doe}
%\blindtext
%\end{IEEEbiography}

% You can push biographies down or up by placing
% a \vfill before or after them. The appropriate
% use of \vfill depends on what kind of text is
% on the last page and whether or not the columns
% are being equalized.

%\vfill

% Can be used to pull up biographies so that the bottom of the last one
% is flush with the other column.
%\enlargethispage{-5in}



% that's all folks
\end{document}


